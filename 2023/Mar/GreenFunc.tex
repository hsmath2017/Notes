\documentclass{article}
%\usepackage{ctex}
\usepackage{amsmath}
\usepackage{amsthm}
\usepackage{geometry}
\geometry{left=2.5cm,right=2.5cm,top=2.5cm,bottom=2.5cm}
\usepackage{amssymb}
\usepackage{indentfirst}
\usepackage{graphicx}
\usepackage{subfigure}
\usepackage{listings}
\usepackage{xcolor}
\usepackage{float}
%\usepackage{algorithm}  
%\usepackage{algorithmicx} 
\usepackage[ruled,linesnumbered]{algorithm2e} 
\usepackage{longtable}
\usepackage{fancyhdr}
\usepackage{appendix}
\usepackage{enumitem}
\usepackage{abstract}
\usepackage{multirow}
\pagestyle{fancy}
\lfoot{}%这条语句可以让页码出现在下方
\theoremstyle{plain}
\newtheorem{thm}{Theorem}
\newtheorem{lem}[thm]{Lemma}
\newtheorem{prop}[thm]{Proposition}
\newtheorem{cor}[thm]{Corollary}

\theoremstyle{definition}
\newtheorem{defn}{Definition}[section]

\theoremstyle{remark}
\newtheorem*{rem}{Remark}
\newtheorem{eg}{Example}[section]
\newcommand{\supp}{\text{supp}}
\newcommand{\Rn}{\mathbb{R}^{n}}
\newcommand{\dif}{\mathrm{d}}
\newcommand{\avg}[1]{\left\langle #1 \right\rangle}
\newcommand{\difFrac}[2]{\frac{\dif #1}{\dif #2}}
\newcommand{\pdfFrac}[2]{\frac{\partial #1}{\partial #2}}
\newcommand{\OFL}{\mathrm{OFL}}
\newcommand{\UFL}{\mathrm{UFL}}
\newcommand{\fl}{\mathrm{fl}}
\newcommand{\op}{\odot}
\newcommand{\cp}{\cdot}
\newcommand{\Eabs}{E_{\mathrm{abs}}}
\newcommand{\Erel}{E_{\mathrm{rel}}}
\newcommand{\DR}{\mathcal{D}_{\widetilde{LN}}^{n}}
\newcommand{\add}[2]{\sum_{#1=1}^{#2}}
\newcommand{\innerprod}[2]{\left<#1,#2\right>}
\newcommand\tbbint{{-\mkern -16mu\int}}
\newcommand\tbint{{\mathchar '26\mkern -14mu\int}}
\newcommand\dbbint{{-\mkern -19mu\int}}
\newcommand\dbint{{\mathchar '26\mkern -18mu\int}}
\newcommand\bint{
{\mathchoice{\dbint}{\tbint}{\tbint}{\tbint}}
}
\newcommand\bbint{
{\mathchoice{\dbbint}{\tbbint}{\tbbint}{\tbbint}}
}
\title{Green Function And Discretized Laplacian Operator}
\author{Shuang Hu}
\date{2023.3.4}
\begin{document}
\maketitle
\section*{Statement}
I would like to acknowledge the help from Huiteng Li.
\section{Introduction}
\begin{defn}
For a fixed $\bar{x}\in [0,1]$, the \textbf{Green's function} $G(x;\bar{x})$ 
is the function of $x$ that solves the BVP
\begin{equation}
    \label{GreenFunc_BVP}
    \left\{
        \begin{aligned}
        &u''(x)=\delta(x-\bar{x});\\
        &u(0)=u(1)=0,\\
        \end{aligned}
    \right.
\end{equation}
where $\delta(x-\bar{x})$ is the Dirac delta function.
\end{defn}
Discretize equation \eqref{GreenFunc_BVP}, then find the inverse of matrix
\begin{equation}
    \label{Discretized_Laplacian}
    A=\frac{1}{h^2}\begin{bmatrix}
        2&-1&&&&\\
        -1&2&-1&&&\\
        &\ddots&\ddots&\ddots&&\\
        &&-1&2&-1&\\
        &&&-1&2&\\
    \end{bmatrix}
\end{equation}
by Green's function. In \eqref{Discretized_Laplacian}, for $n\times n$ matrix $A$, $h=\frac{1}{n+1}$. $A$ is \textbf{discretized Laplacian operator}. 
\section{Discretize the dirac delta function}
Dirac delta function isn't a usual continuous function, so we should find a new method to discretize it. In fact, when we discretize a function, we exactly find a \textbf{grid function} to approximate it. For delta function $\delta(x-\bar{x})$, it satisfies:
\begin{itemize}
    \item $\forall x\neq \bar{x}$, $\delta(x-\bar{x})=0$.
    \item $\int_{-\infty}^{+\infty}\delta(x-\bar{x})\dif x=1$.
\end{itemize}
So, assume $\bar{x}=x_{j}$ is on the discretized grid set $X:=\left\{x_{j}\right\}_{j=1}^{n}$, the discretized grid function $\delta_{g}:X\rightarrow\mathbb{R}$ satisfies:
\begin{itemize}
    \item $\delta_{g}(x_{i})=\delta_{ij}$ while $\delta_{ij}$ is the Kronecker symbol.
    \item $\|\delta_{g}\|_{1}=1$, $\|\cdot\|$ means the \textbf{q-norm} of grid function.
\end{itemize}
Then:
\begin{equation}
    \label{eq:discretized_delta}
    \delta_{g}(x_{i})=\left\{
        \begin{aligned}
        &0, i\neq j;\\
        &\frac{1}{h}, i=j.\\
        \end{aligned}
    \right.
\end{equation}
So, for $\bar{x}=x_{j}$, the discretized linear system for equation \eqref{GreenFunc_BVP} is:
\begin{equation}
    \label{eq:discretized_green_pde}
    -AU=\frac{1}{h}e_{j}.
\end{equation}
When $h\rightarrow 0$, $\delta_{g}\rightarrow \delta$ and $A\rightarrow -\Delta$, so we can use \eqref{eq:discretized_green_pde} to discretize equation \eqref{GreenFunc_BVP}.
\section{Inverse of discretized Laplacian operator}

If we admit \eqref{eq:discretized_green_pde} has no truncated error, we can use green's function to derive the inverse of matrix $A$. For $B=A^{-1}$, mark $B=[b_{1},\cdots,b_{n}]$, we can see:
\begin{equation}
    Ab_{j}=e_{j}.
\end{equation}
If $U_{j}$ satisfies \eqref{eq:discretized_green_pde}, $b_{j}=-h U_{j}$.

By the definition of green's function, when $\bar{x}=x_{j}$, the solution of equation \eqref{GreenFunc_BVP} is $G(x;x_{j})$. Its restriction operator on $X$ is $G(x_{i};x_{j})$. So, if truncated error $\tau=0$, we can see: 
\begin{equation}
    U_{j}=\begin{bmatrix}
        G(x_{1};x_{j})\\
        G(x_{2};x_{j})\\
        \vdots\\
        G(x_{n};x_{j})\\
    \end{bmatrix}.
\end{equation}
So, if $B=A^{-1}$, we can see $b_{ij}=-hG(x_{i};x_{j})$. We can verify that $B$ is indeed the inverse of $A$.
\section{Discussion}
How to show \eqref{eq:discretized_green_pde} has no truncated error?
\end{document}